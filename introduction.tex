\section{INTRODUCTION}
\subsection{Motivation}

To enable the use of drones in industrial inspection, a level of automation must be achieved that ensures an acceptable low probability of collision, as this might harm the complex that is inspected and will hinder continuation of the inspection. A reasonable plan for a path that the drone can follow could be made offline based on 3D models or blueprints of the plant. Directly following this path does not fulfill the requirements of safety as the model might be inaccurate, or the world might have changed since the model was made. Collision avoidance strategies based on on-board percepting sensors must be utilized to achieve an acceptable safety level for the mission.

For small rotary wing drones minimizing the weight of the payload is of great concern. With higher weight comes higher energy consumption which will reduce the operational time of a drone before it has to land and recharge. Advanced precepting sensors such as LIDAR, 2D and 3D cameras are relatively heavy for these drones. Enabling drones using simpler low weight sensors to approach a level of automation that is normally obtained with more advanced high weight sensors, could significantly increase the operational time of the drone. Simple stationary radar sensor fulfills the requirement of reduced weight. It also introduces the advantage of being a too strict sensor that rather gives a pessimistic estimate than a too optimistic one that image recognition often can lead to. A radar will also have larger field of view than a 2D LIDAR, reducing the chance that the drone will hit an unobserved obstacle. Using Radar sensors will therefor give an advantage in weight reduction and insurance that the probability of not observing an obstacle is low. Radar sensors have the disadvantage of giving results with very low resolution making it hard to distinguish safe areas close to obstacles from the obstacles themselves. 

Er en usikkerhet i estimatet tilstand vi bruekr i kontrollere som vil gi en usikkerhet i possisjon når vi simulerer frem i tid. 
Spesielt innendørs er dette problematisk?

    Enkle radar sensorer som har lav vekt
    Usikkerhet i målingene 
    
    Noe filosofi runt at det er umulig å garantere ikke kollisjon, men vi må sette en øvre grense på akseptert risiko.

%\subsection{Literature review}

\subsection{Contributions}
